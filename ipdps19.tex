% Created 2018-09-30 Sun 19:15
% Intended LaTeX compiler: pdflatex
\documentclass[conference]{IEEEtran}

\usepackage{graphicx}
\usepackage{amssymb}
\usepackage{amsmath}
\usepackage{xcolor}
\usepackage{url}
\usepackage{listings}
\usepackage[utf8]{inputenc}
\usepackage[english]{babel}
\usepackage{multirow}
\usepackage{textcomp}
\usepackage{caption}
\usepackage{hyperref}
\usepackage{booktabs}
\usepackage{array}
\usepackage{relsize}
\usepackage{bm}
\usepackage{wasysym}
\usepackage{ragged2e}
\renewcommand*{\UrlFont}{\ttfamily\smaller\relax}
\author{\IEEEauthorblockN{Pedro Bruel\IEEEauthorrefmark{1}\IEEEauthorrefmark{2},
Arnaud Legrand\IEEEauthorrefmark{1},
Brice Videau\IEEEauthorrefmark{1} and
Alfredo Goldman\IEEEauthorrefmark{2}}
\IEEEauthorblockA{\IEEEauthorrefmark{1}Univ. Grenoble Alpes, CNRS, INRIA, LIG - Grenoble, France\\
Email: \{arnaud.legrand, brice.videau\}@imag.fr}
\IEEEauthorblockA{\IEEEauthorrefmark{2}Univ. of São Paulo - São Paulo, Brazil\\
Email: \{phrb,gold\}@ime.usp.br}}
\date{\today}
\title{Autotuning under Budget Constraints: a Design of Experiments Approach}
\hypersetup{
 pdfauthor={},
 pdftitle={Autotuning under Budget Constraints: a Design of Experiments Approach},
 pdfkeywords={},
 pdfsubject={},
 pdfcreator={Emacs 26.1 (Org mode 9.1.14)},
 pdflang={English}}
\begin{document}

\maketitle
\begin{abstract}
Abstract
\end{abstract}

\section{Introduction}
\label{sec:orgaa3a372}
Optimizing code for objectives such as performance and power consumption is
fundamental to the success and cost effectiveness of industrial and scientific
endeavours related to High Performance Computing. A considerable amount of
highly specialized time and effort is spent in porting code to GPUs, FPGAs and
other hardware accelerators. Experts are also needed in leveraging bleeding edge
software improvements in compilers, languages, libraries and frameworks.

The automatic configuration and optimization of High Performance Computing
applications, or autotuning, decreases the cost and time needed to adopt
efficient hardware and sofware. Typical targets for autotuning include algorithm
selection, source-to-source transformations and compiler configuration.

The autotuning of Hgh Performance Computing applications can be studied as a
search problem, where the objective is to minimize single or multiple sofware of
hardware metrics. The exploration of the search spaces defined by configurations
and optimizations present interesting challenges to search algorithms. These
search spaces grow exponentially with the number of considered configuration
parameters and their possible values, and it is also difficult to explore these
search spaces extensively, due to the often prohibitive costs of hardware
utilization and program execution times. Developing efficient autotuning
strategies is therefore essential for decreasing optimization cost and time.

\section{Related Work}
\label{sec:org794e81c}
\subsection{Source-to-source Transformation}
\label{sec:org1dd77b5}
\subsection{Autotuning}
\label{sec:orgebe1a2b}
\subsection{Search Space Exploration Strategies}
\label{sec:orgc7165d4}

\section{Design of Experiments}
\label{sec:org1a508b1}
\subsection{D-Optimal Designs}
\label{sec:orgb65403c}

\section{Applying Design of Experiments to Autotuning}
\label{sec:org7e7566d}

\subsection{Experimental Methodology}
\label{sec:org7d30f97}
\subsection{Performance on a GPU Laplacian Kernel}
\label{sec:orgca83fb2}

\section{Results on the SPAPT Benchmark}
\label{sec:org2ebd0ab}

\section{Conclusion}
\label{sec:org5e5218a}
\end{document}
