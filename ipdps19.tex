% Created 2018-10-15 Mon 18:06
% Intended LaTeX compiler: pdflatex
\documentclass[conference]{IEEEtran}

\usepackage{graphicx}
\usepackage{amssymb}
\usepackage{amsmath}
\usepackage{colortbl}
\usepackage{xcolor}
\usepackage{url}
\usepackage{listings}
%\usepackage[utf8]{inputenc}
\usepackage[english]{babel}
\usepackage{multirow}
\usepackage{caption}
\usepackage{hyperref}
\usepackage{booktabs}
\usepackage{array}
\usepackage{relsize}
\usepackage{bm}
\usepackage{wasysym}
\usepackage{ragged2e}
\usepackage{todonotes}
\usepackage{tabularx}
\usepackage{boxedminipage}
\usepackage[all]{nowidow}
\lstset{ %
backgroundcolor={},
basicstyle=\ttfamily\scriptsize,
breakatwhitespace=true,
breaklines=true,
captionpos=n,
extendedchars=true,
frame=n,
rulecolor=\color{black},
showspaces=false,
showstringspaces=false,
showtabs=false,
stepnumber=2,
stringstyle=\color{gray},
tabsize=2,
}
\renewcommand*{\UrlFont}{\ttfamily\smaller\relax}
\makeatletter
\def\maketag@@@#1{\hbox{\m@th\normalfont\normalsize#1}}
\makeatother
\graphicspath{{./img/}}
\renewcommand*{\UrlFont}{\ttfamily\smaller\relax}
\author{\IEEEauthorblockN{\scalebox{.95}{Pedro Bruel\IEEEauthorrefmark{1}\IEEEauthorrefmark{2},
Steven Quinito Masnada\IEEEauthorrefmark{3},
Brice Videau\IEEEauthorrefmark{1},
Arnaud Legrand\IEEEauthorrefmark{1},
Jean-Marc Vincent\IEEEauthorrefmark{1},
Alfredo Goldman\IEEEauthorrefmark{2}}}
\smallskip
\IEEEauthorblockA{\begin{minipage}[t]{.2\linewidth}\centering\IEEEauthorrefmark{2}University of São Paulo \\ São Paulo, Brazil\\
\small\{phrb, gold\}@ime.usp.br\null\vspace{-15pt}\end{minipage}\hfill
\begin{minipage}[t]{.28\linewidth}\centering \IEEEauthorrefmark{3}University of Grenoble Alpes \\ Inria, CNRS, Grenoble INP, LJK \\ 38000 Grenoble, France\\
\small steven.quinito-masnada@inria.fr\null\vspace{-15pt}\end{minipage}\hfill
\begin{minipage}[t]{.45\linewidth}\centering\IEEEauthorrefmark{1}University of Grenoble Alpes \\ CNRS, Inria, Grenoble INP, LIG \\ 38000 Grenoble, France\\
\small\{arnaud.legrand, brice.videau, jean-marc.vincent\}@imag.fr\null\vspace{-15pt}\end{minipage}}}
\date{\today}
\title{Autotuning under Tight Budget Constraints:  \\ A Transparent Design of Experiments Approach}
\hypersetup{
 pdfauthor={},
 pdftitle={Autotuning under Tight Budget Constraints:  \\ A Transparent Design of Experiments Approach},
 pdfkeywords={},
 pdfsubject={},
 pdfcreator={Emacs 26.1 (Org mode 9.1.14)},
 pdflang={English}}
\begin{document}

\maketitle
\begin{abstract}
A large amount of resources is spent writing, porting, and optimizing scientific
and industrial High Performance Computing applications, which makes autotuning
techniques fundamental to lower the cost of leveraging the improvements on
execution time and power consumption provided by the latest software and
hardware platforms. Despite the need for economy, most autotuning techniques
still require a large budget of costly experimental measurements to provide good
results, while rarely providing exploitable knowledge after optimization. The
contribution of this paper is a user-transparent autotuning technique based on
Design of Experiments that operates under tight budget constraints by
significantly reducing the measurements needed to find good optimizations. Our
approach enables users to make informed decisions on which optimizations to
pursue and when to stop. We present an experimental evaluation of our approach
and show it is capable of leveraging user decisions to find the best global
configuration of a GPU Laplacian kernel using half of the measurement budget
used by other common autotuning techniques. We show that our approach is also
capable of finding speedups of up to \boldmath\(50\times\), compared to \texttt{gcc}'s
\texttt{-O3}, for some kernels from the SPAPT benchmark, using up to
\boldmath\(10\times\) less measurements than random sampling.
\end{abstract}

\section{Introduction}
\label{sec:org77052f3}
Optimizing code for objectives such as performance and power consumption is
fundamental to the success and cost effectiveness of industrial and scientific
endeavors in High Performance Computing. A considerable amount of highly
specialized time and effort is spent in porting and optimizing code for GPUs,
FPGAs and other hardware accelerators. Experts are also needed to leverage
bleeding edge software improvements in compilers, languages, libraries and
frameworks. The objective of techniques for the automatic configuration and
optimization of High Performance Computing applications, or \emph{autotuning}, is to
decrease the cost and time needed to adopt efficient hardware and software.
Typical autotuning targets include algorithm selection, source-to-source
transformations and compiler configuration.

Autotuning can be studied as a search problem where the objective is to minimize
software of hardware metrics. The exploration of the search spaces defined by
code and compiler configurations and optimizations presents interesting
challenges. These search spaces grow exponentially with the number of parameters
and their possible values. They are also difficult to extensively explore due to
the often prohibitive costs of hardware utilization, program compilation and
execution times. Developing autotuning strategies capable of producing good
optimizations while minimizing resource utilization is therefore essential. The
capability of acquiring knowledge about an optimization problem is also a
desired feature of an autotuning strategy, since this knowledge can decrease the
cost of subsequent optimizations of the same application or for the same
hardware.

It is common and usually effective to use search meta-heuristics such as genetic
algorithms and simulated annealing in autotuning. These strategies attempt to
exploit local search space properties, but are generally incapable of exploiting
global structures. Seymour \emph{et al.}~\cite{seymour2008comparison},
Knijnenburg \emph{et al.}~\cite{knijnenburg2003combined}, and Balaprakash \emph{et
al.}~\cite{balaprakash2011can,balaprakash2012experimental} report that
these strategies are not more effective than a naive uniform random sample of
the search space, and usually rely on a large number of measurements and
frequent restarts to achieve good performance improvements. Search strategies
based on gradient descent are also commonly used in autotuning and also rely on
a large number of measurements. Their effectiveness diminishes significantly in
search spaces with complex local structures. Automated machine learning
autotuning
strategies~\cite{beckingsale2017apollo,falch2017machine,balaprakash2016automomml}
are promising in building models for predicting important optimization
parameters, but still rely on a sizable data set for training.

Search strategies based on meta-heuristics, gradient descent and machine
learning require a large number of measurements to be effective, and are usually
incapable of providing knowledge about search spaces to users. Since these
strategies are not transparent, at the end of each autotuning session it is
difficult to decide if and where further exploration is warranted, and often
impossible to know which parameters are responsible for the observed
improvements. After exploring a search space, it is impossible to confidently
deduce its global properties since its was automatically explored with unknown
biases.

The contribution of this paper is an autotuning strategy that leverages existing
knowledge about a problem by using a initial performance model, and refining it
iteratively using performance measurements, statistical analysis, and user
input. Our strategy places a heavy weight on decreasing autotuning costs by
using a \emph{Design of Experiments} methodology to minimize the number of
experiments needed to find good optimizations. Each iteration uses \emph{Analysis of
Variance} (ANOVA) tests and \emph{linear model regressions} to identify promising
subspaces and the relative significance of each parameter to observations. An
architecture- and problem-specific performance model is built iteratively and
with user input, which enables making informed decisions about which regions of the
search space are worth exploring.

We evaluate the performance of our approach on optimizing a Laplacian Kernel for
GPUs, where the search space, the global optimum and a performance model
approximation are known. The budget of measurements was tightly constrained on
this experiment. Speedups and budget utilization reduction achieved by our
approach on this setting motivated a more comprehensive performance evaluation.
We chose the \emph{Search Problems in Automatic Performance Tuning}
(SPAPT)~\cite{balaprakash2012spapt} benchmark suite for this evaluation,
where we obtained diverse results. Out of the 17 SPAPT kernels benchmarked, no
speedup could be found for 3 kernels, and random sampling was very effective for
the others. For the remaining 8 kernels our approach found speedups of up to
\(50\times\), compared to \texttt{gcc}'s \texttt{-O3} without code transformation, while using
up to \(10\times\) less measurements than random sampling.

The rest of this paper is organized as follows. Section~\ref{sec:org78a0f49}
presents related work on source-to-source transformation, which is the main
optimization target in SPAPT kernels, on autotuning systems and on search space
exploration strategies. Section~\ref{sec:org7e9f5c3} discusses the Design
of Experiments, ANOVA, and linear regression methodology we used to develop our
approach. Section~\ref{sec:org2d608ec} discusses the
implementation of our approach in detail. Section~\ref{sec:org52ac7d9}
presents the results on the GPU Laplacian Kernel and on the SPAPT benchmark
suite. Section~\ref{sec:org169f53c} discusses our conclusions and future work.
\section{Background}
\label{sec:org78a0f49}
This Section presents the background and related work on source-to-source
transformation, autotuning systems and search space exploration strategies.
\subsection{Source-to-Source Transformation}
\label{sec:org9a35410}
Our approach can be applied to any autotuning domain that expresses optimization
as a search problem, although the performance evaluations we present in
Section~\ref{sec:org52ac7d9} were obtained in the domain of
source-to-source transformation. Several frameworks, compilers and autotuners
provide tools to generate and optimize architecture-specific
code~\cite{hartono2009annotation,videau2017boast,tiwari2009scalable,yi2007poet,ansel2009petabricks}.
We used BOAST~\cite{videau2017boast} and
Orio~\cite{hartono2009annotation} to perform source-to-source
transformations targeting parallelization on CPUs and GPUs, vectorization, loop
transformations such as tiling and unrolling, and data structure size and
copying.
\subsection{Autotuning}
\label{sec:org5a1f84c}
John Rice's Algorithm Selection framework~\cite{rice1976algorithm} is the
precursor of autotuners in various problem domains. In 1997, the PHiPAC
system~\cite{bilmes1997optimizing} used code generators and search scripts
to automatically generate high performance code for matrix multiplication. Since
then, systems approached different domains with a variety of strategies.
Dongarra \emph{et al.}~\cite{dongarra1998automatically} introduced the ATLAS
project, that optimizes dense matrix multiplication routines. The
OSKI~\cite{vuduc2005oski} library provides automatically tuned kernels for
sparse matrices. The FFTW~\cite{frigo1998fftw} library provides tuned C
subroutines for computing the Discrete Fourier Transform.
Periscope~\cite{gerndt2010automatic} is a distributed online autotuner for
parallel systems and single-node performance. In an effort to provide a common
representation of multiple parallel programming models, the INSIEME compiler
project~\cite{jordan2012multi} implements abstractions for OpenMP, MPI and
OpenCL, and generates optimized parallel code for heterogeneous multi-core
architectures.

A different approach is to combine generic search algorithms and problem
representation data structures in a single system that enables the
implementation of autotuners for different domains. The
PetaBricks~\cite{ansel2009petabricks} project provides a language,
compiler and autotuner, enabling the definition and selection of multiple
algorithms for the same problem. The ParamILS
framework~\cite{hutter2009paramils} applies stochastic local search
algorithms to algorithm configuration and parameter tuning. The OpenTuner
framework~\cite{ansel2014opentuner} provides ensembles of techniques that
search the same space in parallel, while exploration is managed by a multi-armed
bandit strategy.
\subsection{Search Space Exploration Strategies}
\label{sec:org4b4ae2c}
\begin{figure}[htbp]
\centering
\includegraphics[width=.95\columnwidth]{./img/sampling_comparison.pdf}
\caption{\label{fig:orgc3cf1e1}
Exploration of the search space, using a fixed budget of 50 points. The red ``\(+\)'' represents the best point found by each strategy, and ``\(\times\)''s denote neighborhood exploration}
\end{figure}

Figure~\ref{fig:orgc3cf1e1} shows the contour of a search space defined
by a function of the form \(z = x^2 + y^2 + \varepsilon\), where \(\varepsilon\) is
a local perturbation, and the exploration of that search space by six different
strategies. In such a simple search space, even a uniform random sample can find
points close to the optimum, despite not exploiting geometry. A Latin
Hypercube~\cite{carnell2018lhs} sampling strategy covers the search space
more evenly, but still does not exploit its geometry. Strategies based on
neighborhood exploration such as simulated annealing and gradient descent can
exploit local structures, but may get trapped in local minima. Their
performance is strongly dependent on search starting point. These strategies do
not leverage global search space structure, or provide exploitable knowledge
after optimization.

Measurement of the kernels optimized on the performance evaluations in
Section~\ref{sec:org52ac7d9} can exceed 20 minutes, including the time
of code transformation, compilation, and execution. Measurements in other
problem domains can take much longer to complete. This strengthens the motivation
to consider search space exploration strategies capable of operating under tight
budget constraints. These strategies have been developed and improved by
statisticians for a long time, and can be grouped under the Design of
Experiments term.

The D-Optimal sampling strategies shown on the two rightmost bottom panels of
Figure~\ref{fig:orgc3cf1e1} are based on the Design of Experiments
methodology, and leverage previous knowledge about search spaces for an
efficient exploration. These strategies provide transparent analyses that
enable focusing on interesting subspaces. In the next
Section we present the Design of Experiments methodology used to implement our
approach.
\section{Design of Experiments}
\label{sec:org7e9f5c3}
An \emph{experimental design} determines a selection of experiments whose objective
is to identify the relationships between \emph{factors} and \emph{responses}. While
factors and responses can refer to different concrete entities in other domains,
in computer experiments factors can be configuration parameters for algorithms
and compilers, for example, and responses can be the execution time or memory
consumption of a program. Each possible value of a factor is called a \emph{level}.
The \emph{effect} of a factor on the measured response, without its \emph{interactions}
with other factors, is the \emph{main effect} of that factor. Experimental designs
can be constructed with different goals, such as identifying the main effects
or building an analytical model for the response.

In this Section we first present the assumptions of a traditional Design of
Experiments methodology using an example of \emph{2-level screening designs}, which
are an efficient way to identify main effects. We then discuss some techniques
for the construction of efficient designs for factors with arbitrary numbers and
types of levels, and present \emph{D-Optimal} designs, the technique we use in the
approach presented in this paper.
\subsection{Screening \& Plackett-Burman Designs}
\label{sec:orgd5313a5}
Screening designs provide a parsimonious way to identify the main
effects of 2-level factors in the initial stages of studying a problem. While
interactions are not considered at this stage, identifying main effects early
enables focusing on a smaller set of factors on subsequent more detailed
experiments. A specially efficient design construction technique for screening
designs was presented by Plackett and Burman~\cite{plackett1946design} in
1946, and is available in the \texttt{FrF2}
package~\cite{gromping2014frf2} of the \texttt{R}
language~\cite{team2018rlanguage}.

Despite having strong restrictions on the number of factors they support,
Plackett-Burman designs enable the identification of main effects of \(n\) factors
with \(n + 1\) experiments. Factors may have many levels, but Plackett-Burman
designs can only be constructed for 2-level factors. Therefore, before
constructing a Plackett-Burman design we must identify \emph{high} and \emph{low} levels
for each factor.

Assuming a crude linear relationship between factors and the response is
fundamental for running ANOVA tests using a Plackett-Burman design. Consider the
following linear relationship:
\vspace{-5pt}
{\normalsize
\begin{align}
\mathbf{Y} = \bm{\beta}\mathbf{X} + \varepsilon,
%\caption{Linear model assumed in main-effect analysis of screening designs}
\label{eq:linear_assumption}
\end{align}
}
\vspace{-1pt}
where \(\varepsilon\) is the error term, \(\mathbf{Y}\) is the observed response,
\(\mathbf{X} = \left\{1, x_1,\dots,x_n\right\}\) is the set of \(n\) 2-level
factors, and \(\bm{\beta} = \left\{\beta_0,\dots,\beta_n\right\}\) is the set with
the \emph{intercept} \(\beta_0\) and the corresponding \emph{model coefficients}. ANOVA
tests can rigorously compute the significance of each factor. We can think of
that intuitively by noting that less relevant factors will have corresponding
values in \(\bm{\beta}\) close to zero.

We now present an example to illustrate the screening methodology. Suppose we
wish to minimize a performance metric \(Y\) of a problem with factors
\(x_1,\dots,x_8\) assuming values are in \(\left\{-1, -0.8, -0.6, \dots, 0.6, 0.8,
1\right\}\). Each \(y_i \in Y\) is computed using the following equation:
\vspace{-5pt}
{\normalsize
\begin{align}
\label{eq:real_model}
y_i = & -1.5x_1 + 1.3x_3 + 3.1x_5 + \\
& -1.4x_7 + 1.35x_8^2 + 1.6x_3x_5 + \varepsilon. \nonumber
\end{align}
%\caption{Real model used to obtain the data on Table\ref{tab:plackett}}
}
\vspace{-1pt}
Suppose that, for the purpose of this example, they are computed by a very
expensive black-box procedure. Note that factors \(\{x_2,x_4,x_6\}\) have no
influence. In this scenario we can think of the error term \(\varepsilon\) as
representing not only noise but our uncertainty regarding the model as well.
Higher amplitudes of \(\varepsilon\) might make it harder to justify isolating
factors with low significance.

To efficiently study this problem we decide to construct a Plackett-Burman
design, which minimizes the experiments needed to identify relevant factors. The
analysis of this design will enable us to decrease the dimension of the problem.
Table~\ref{tab:plackett} presents the Plackett-Burman design we generated.
It contains high and low values, chosen to be \(-1\) and \(1\), for the factors
\(x_1,\dots,x_8\), and the observed response \(\mathbf{Y}\). As is common when
constructing screening designs, we had to add 3 ``dummy'' factors
\(d_1,\dots,d_3\) to complete the 12 columns needed to construct a Plackett-Burman
design for 8 factors.

% latex table generated in R 3.5.1 by xtable 1.8-2 package
% Mon Oct 15 18:04:00 2018
\begin{table}[b]
\centering
\caption{Randomized Plackett-Burman design for factors $x_1, \dots, x_8$, using 12 experiments and ``dummy'' factors $d_1, \dots, d_3$, and computed response $\mathbf{Y}$}
\label{tab:plackett}
\begingroup\scriptsize
\begin{tabular}{cccccccccccc}
  \toprule
$x_1$ & $x_2$ & $x_3$ & $x_4$ & $x_5$ & $x_6$ & $x_7$ & $x_8$ & $d_1$ & $d_2$ & $d_3$ & $Y$ \\
  \midrule
1 & -1 & 1 & 1 & 1 & -1 & -1 & -1 & 1 & -1 & 1 & 13.74 \\
  -1 & 1 & -1 & 1 & 1 & -1 & 1 & 1 & 1 & -1 & -1 & 10.19 \\
  -1 & 1 & 1 & -1 & 1 & 1 & 1 & -1 & -1 & -1 & 1 & 9.22 \\
  1 & 1 & -1 & 1 & 1 & 1 & -1 & -1 & -1 & 1 & -1 & 7.64 \\
  1 & 1 & 1 & -1 & -1 & -1 & 1 & -1 & 1 & 1 & -1 & 8.63 \\
  -1 & 1 & 1 & 1 & -1 & -1 & -1 & 1 & -1 & 1 & 1 & 11.53 \\
  -1 & -1 & -1 & 1 & -1 & 1 & 1 & -1 & 1 & 1 & 1 & 2.09 \\
  1 & 1 & -1 & -1 & -1 & 1 & -1 & 1 & 1 & -1 & 1 & 9.02 \\
  1 & -1 & -1 & -1 & 1 & -1 & 1 & 1 & -1 & 1 & 1 & 10.68 \\
  1 & -1 & 1 & 1 & -1 & 1 & 1 & 1 & -1 & -1 & -1 & 11.23 \\
  -1 & -1 & -1 & -1 & -1 & -1 & -1 & -1 & -1 & -1 & -1 & 5.33 \\
  -1 & -1 & 1 & -1 & 1 & 1 & -1 & 1 & 1 & 1 & -1 & 14.79 \\
   \bottomrule
\end{tabular}
\endgroup
\end{table}

We use our initial assumption shown in Equation~\eqref{eq:linear_assumption} to
identify the most relevant factors by performing an ANOVA test. The resulting
ANOVA table is shown in Table~\ref{tab:anova_linear}, where the \emph{significance}
of each factor can be interpreted from the F-test and P\((<\text{F})\) values.
Table~\ref{tab:anova_linear} uses ``\(*\)'', as is convention in the \texttt{R}
language, to represent the significance values for each factor.

We see on Table~\ref{tab:anova_linear} that factors
\(\left\{x_3,x_5,x_7,x_8\right\}\) have at least one ``\(*\)'' of significance. For
the purpose of this example, this is sufficient reason to include them in our
linear model for the next step. We decide as well to discard factors
\(\left\{x_2,x_4,x_6\right\}\) in our model for now, due to their low
significance. We see that factor \(x_1\) has a significance mark of ``\(\cdot\)'', but
comparing its F-test and P\((<\text{F})\) values we decide that they are fairly
smaller than the values of factors that had no significance at all, and we keep
this factor.

% latex table generated in R 3.5.1 by xtable 1.8-2 package
% Mon Oct 15 18:04:10 2018
\begin{table}[t]
\centering
\caption{Shortened ANOVA table for the fit of the naive model, with significance intervals from the \texttt{R} language}
\label{tab:anova_linear}
\begingroup\small
\begin{tabular}{lrrl}
  \toprule
 & F value & Pr$(<\text{F})$ & Signif. \\
  \midrule
$x_1$ & 8.382 & 0.063 & $\cdot$ \\
  $x_2$ & 0.370 & 0.586 &   \\
  $x_3$ & 80.902 & 0.003 & $**$ \\
  $x_4$ & 0.215 & 0.675 &   \\
  $x_5$ & 46.848 & 0.006 & $**$ \\
  $x_6$ & 5.154 & 0.108 &   \\
  $x_7$ & 13.831 & 0.034 & $*$ \\
  $x_8$ & 59.768 & 0.004 & $**$ \\
   \bottomrule
\end{tabular}
\endgroup
\end{table}

Moving forward, we will build a linear model using factors
\(\left\{x_1,x_3,x_5,x_7,x_8\right\}\), fit the model using the values of \(Y\) we
obtained when running our design, and use the coefficients of this fitted model
to predict the levels for each factor that minimize the real response. We can do
that because these factors are numerical, even though only discrete values are
allowed.

We now proceed to the prediction step, where we wish to identify the levels of
factors \(\left\{x_1,x_3,x_5,x_7,x_8\right\}\) that minimize our fitted model,
without running any new experiments. This can be done by, for example, using a
gradient descent algorithm or finding the point where the derivative of the
function given by the linear regression equals to zero.

Table~\ref{tab:linear_prediction_comparison} compares the prediction for
\(Y\) from our linear model with the selected factors
\(\left\{x_1,x_3,x_5,x_7,x_8\right\}\) with the actual global minimum \(Y\) for this
problem. Note that factors \(\left\{x_2,x_4,x_6\right\}\) are included for the
global minimum. This happens here because of the error term \(\varepsilon\),
but could also be interpreted as due to model uncertainty.

% latex table generated in R 3.5.1 by xtable 1.8-2 package
% Mon Oct 15 18:04:58 2018
\begin{table}[b]
\centering
\caption{Comparison of the response $Y$ predicted by the linear model and the true global minimum. Factors used in the model are bolded}
\label{tab:linear_prediction_comparison}
\begingroup\footnotesize
\begin{tabularx}{\columnwidth}{lrlrlrlrrr}
  \toprule
 & $\bm{x_1}$ & $x_2$ & $\bm{x_3}$ & $x_4$ & $\bm{x_5}$ & $x_6$ & $\bm{x_7$} & $\bm{x_8}$ & $Y$ \\
  \midrule
 Lin. & -1.0 & -- & -1.0 & -- & -1.0 & -- & 1.0 & -1.0 & -1.046 \\
  Min. & 1.0 & -0.2 & -1.0 & 0.6 & -1.0 & 0.4 & 0.8 & 0.0 & -9.934 \\
   \bottomrule
\end{tabularx}
\endgroup
\end{table}

Using 12 measurements and a simple linear model, the predicted best
value of \(Y\) was around \(10\times\) larger than the global optimum. Note that the
model predicted the correct levels for \(x_3\) and \(x_5\), and almost predicted
correctly for \(x_7\). The linear model predicted wrong levels for \(x_1\), perhaps
due to this factor's interaction with \(x_3\), and for \(x_8\). Arguably, it would
be impossible to predict the correct level for \(x_8\) using this linear model,
since a quadratic term composes the true formula of \(Y\). As we showed in
Figure~\ref{fig:orgc3cf1e1}, a D-Optimal design using a linear model
could detect the significance of a quadratic term, but the resulting
regression will often predict the wrong minimum point.

We can improve upon this result if we introduce some information about the
problem and use a more flexible design construction technique. Next, we will
discuss the construction of efficient designs using problem-specific formulas
and continue the optimization of our example.
\subsection{D-Optimal Designs}
\label{sec:org5fa0ca7}
The application of Design of Experiments to autotuning problems requires design
construction techniques that support factors of arbitrary types and number of
levels. Autotuning problems typically combine factors such as binary flags,
integer and floating point numerical values, and unordered enumerations of
abstract values. Previously, to construct a Plackett-Burman design for our
example we had to restrict our factors to the extremes of their levels in the
interval \(\left\{-1, -0.8, -0.6,\dots,0.6, 0.8, 1\right\}\), because such designs
only support 2-level factors. We have seen that this restriction makes it
difficult to measure the significance of quadratic terms in the model. We will
now show how to further optimize our example by using \emph{D-Optimal designs}, which
increase the number of levels we can efficiently screen for and enables
detecting the significance of more complex model terms.

To construct a D-Optimal design it is necessary to choose an initial model,
which can be done based on previous experiments or on expert knowledge of the
problem. Once a model is selected, algorithmic construction is performed by
searching for the set of experiments that minimizes \emph{D-Optimality}, a measure of
the \emph{variance} of the \emph{estimators} for the \emph{regression coefficients} associated
with the selected model. This search is usually done by swapping experiments
from the current candidate set with experiments from a pool of possible
experiments, according to certain rules, until some stopping criterion is met.
In the example in this section, as well as in the approach presented in this
paper, we use Fedorov's algorithm~\cite{fedorov1972theory} for
constructing D-Optimal designs, implemented in \texttt{R} in the \texttt{AlgDesign}
package~\cite{wheeler2014algdesign}.

Going back to our example, suppose that in addition to using our previous
screening results we decide to hire an expert in our problem's domain. The
expert confirms our initial assumptions that the factor \(x_1\) should be included
in our model since it is usually relevant for this kind of problem and has a
strong interaction with factor \(x_3\). She also mentions we should replace
the linear term for \(x_8\) by a quadratic term for this factor.

Using our previous screening and the domain knowledge provided by our expert, we
choose a new performance model and use it to construct a D-Optimal design using
Fedorov's algorithm. Since we need enough degrees of freedom to fit our model,
we construct the design with 12 experiments shown in Table~\ref{tab:d_optimal}.
Note that the design includes \(-1\), \(0\) and \(1\) levels for factor \(x_8\). The design
will sample from different regions of the search space due to the quadratic term,
as was shown in Figure~\ref{fig:orgc3cf1e1}.

% latex table generated in R 3.5.1 by xtable 1.8-2 package
% Mon Oct 15 18:05:40 2018
\begin{table}[t]
\centering
\caption{D-Optimal design constructed for the factors $\left\{x_1,x_3,x_5,x_7,x_8\right\}$ and computed response $Y$}
\label{tab:d_optimal}
\begingroup\footnotesize
\begin{tabular}{rrrrrr}
  \toprule
$x_1$ & $x_3$ & $x_5$ & $x_7$ & $x_8$ & $Y$ \\
  \midrule
-1.0 & -1.0 & -1.0 & -1.0 & -1.0 & 2.455 \\
  -1.0 & 1.0 & 1.0 & -1.0 & -1.0 & 6.992 \\
  1.0 & -1.0 & -1.0 & 1.0 & -1.0 & -7.776 \\
  1.0 & 1.0 & 1.0 & 1.0 & -1.0 & 4.163 \\
  1.0 & 1.0 & -1.0 & -1.0 & 0.0 & 0.862 \\
  -1.0 & 1.0 & 1.0 & -1.0 & 0.0 & 5.703 \\
  1.0 & -1.0 & -1.0 & 1.0 & 0.0 & -9.019 \\
  -1.0 & -1.0 & 1.0 & 1.0 & 0.0 & 2.653 \\
  -1.0 & -1.0 & -1.0 & -1.0 & 1.0 & 1.951 \\
  1.0 & -1.0 & 1.0 & -1.0 & 1.0 & 0.446 \\
  -1.0 & 1.0 & -1.0 & 1.0 & 1.0 & -2.383 \\
  1.0 & 1.0 & 1.0 & 1.0 & 1.0 & 4.423 \\
   \bottomrule
\end{tabular}
\endgroup
\end{table}

We are now going to fit this model using the results of the experiments in our
D-Optimal design. Table~\ref{tab:correct_fit} shows the model fit table
and compares the estimated and real model coefficients. This example illustrates
that the Design of Experiments approach can achieve close model estimations
using few resources, provided it is able to use user input to identify relevant
factors and knowledge about the problem domain to tweak the model.

% latex table generated in R 3.5.1 by xtable 1.8-2 package
% Mon Oct 15 18:06:01 2018
\begin{table}[ht]
\centering
\caption{Correct model fit comparing real and estimated coefficients, with significance intervals from the \texttt{R} language}
\label{tab:correct_fit}
\begingroup\small
\begin{tabular}{lrrrrl}
  \toprule
 & Real & Estimated & t value & Pr$(>|\text{t}|)$ & Signif. \\
  \midrule
Intercept & 0.000 & 0.050 & 0.305 & 0.776 &   \\
  $x_1$ & -1.500 & -1.452 & -14.542 & 0.000 & *** \\
  $x_3$ & 1.300 & 1.527 & 15.292 & 0.000 & *** \\
  $x_5$ & 3.100 & 2.682 & 26.857 & 0.000 & *** \\
  $x_7$ & -1.400 & -1.712 & -17.141 & 0.000 & *** \\
  $x_8$ & 0.000 & -0.175 & -1.516 & 0.204 &   \\
  $x_8^2$ & 1.350 & 1.234 & 6.180 & 0.003 & ** \\
  $x_1x_3$ & 1.600 & 1.879 & 19.955 & 0.000 & *** \\
   \bottomrule
\end{tabular}
\endgroup
\end{table}

Table~\ref{tab:prediction_comparisons} compares the global minimum in this
example with the predictions made by our initial linear model from the screening
step and our improved model from this step. Using screening, D-Optimal designs,
and domain knowledge we found an optimization within \(10\%\) of the global
optimum computing \(Y\) only 24 times. We were able to do that by first reducing
the dimension of the problem when we eliminated irrelevant factors in the
screening step. We then constructed a more careful exploration of this new
problem subspace, helped by domain knowledge provided by an expert. Note that we
could have reused some of the 12 experiments from the previous step to reduce
the size of the new design even further.

% latex table generated in R 3.5.1 by xtable 1.8-2 package
% Mon Oct 15 18:06:07 2018
\begin{table}[ht]
\centering
\caption{Comparison of the response $Y$ predicted by our models and the true global minimum. Factors used in the models are bolded}
\label{tab:prediction_comparisons}
\begingroup\footnotesize
\begin{tabular}{lrlrlrlrrr}
  \toprule
 & $\bm{x_1}$ & $x_2$ & $\bm{x_3}$ & $x_4$ & $\bm{x_5}$ & $x_6$ & $\bm{x_7$} & $\bm{x_8}$ & $Y$ \\
  \midrule
 Quad. & 1.0 & -- & -1.0 & -- & -1.0 & -- & 1.0 & 0.0 & -9.019 \\
  Lin. & -1.0 & -- & -1.0 & -- & -1.0 & -- & 1.0 & -1.0 & -1.046 \\
  Min. & 1.0 & -0.2 & -1.0 & 0.6 & -1.0 & 0.4 & 0.8 & 0.0 & -9.934 \\
   \bottomrule
\end{tabular}
\endgroup
\end{table}

We are able to explain the performance improvements we obtained in each step of
the process, because we finish steps with a performance model and a performance
prediction. Each factor is included or removed using information obtained in
statistical tests or expert knowledge. If we need to optimize this problem
again, for a different architecture or with larger input, for example, we could
start exploring the search space with a less naive model. We could also continue
the optimization of this problem by further exploring levels of factors
\(\left\{x_2,x_4,x_6\right\}\). The significance of these factors could now be
detectable by ANOVA tests since the other factors are now fixed. If we still
cannot identify any significant factor, it might advisable to spend the
remaining budget using another exploration strategy such as uniform random or
lating hypercube sampling.

The process of screening for factor significance using ANOVA and fitting a
new model using acquired knowledge is essentially a step in the transparent
Design of Experiments approach we present in the next section.
\section{Autotuning with Design of Experiments}
\label{sec:org2d608ec}
In this section we discuss in detail our iterative Design of Experiments
approach to autotuning. At the start of the process it is necessary to define
the factors and levels that compose the search space of the target problem,
select an initial performance model, and generate an experimental design. Then,
as discussed in the previous section, we identify relevant factors by running an
ANOVA test on the results. This enables selecting and fitting a new performance
model, which is used for predicting levels for each relevant factor. The process
can then restart, generating a new design for the new problem subspace with the
remaining factors. Informed decisions made by the user play a central role in
each iteration, guiding and speeding up the process.
Figure~\ref{fig:org2bd2283} presents an overview of our approach.

\begin{figure}[b]\vspace{-.5cm}
\centering
\includegraphics[width=.95\columnwidth]{./img/doe_anova_strategy.pdf}
\caption{\label{fig:org2bd2283}
Overview of the Design of Experiments approach to autotuning proposed in this paper}
\end{figure}

The first step of our approach is to define which are the target factors and
which levels of each factor are worth exploring. Then, the user must select an
initial performance model. Compilers typically expose many 2-level factors in
the form of configuration flags. The performance model for a single flag can
only be a linear term, since there are only 2 values to measure. Interactions
between flags can also be considered in an initial model. Numerical factors are
also common, such as block sizes for CUDA programs or loop unrolling amounts.
Deciding which levels to include for these kinds of factors requires a more
careful analysis. For example, if we suspect the performance model has a
quadratic term for a certain factor, we should include at least three of its
levels. The ordering between the levels of other compiler parameters, such as
\texttt{-O(0,1,2,3)}, is not obviously translated to a number. Factors like
these are named \emph{categorical}, and must be treated differently when constructing
designs and analyzing the results.

We decided to use D-Optimal designs because their construction techniques enable
mixing categorical and numerical factors in the same screening design, while
biasing sampling according to the performance model. This enables the autotuner to
exploit global search space structures if we use the right model. When
constructing a D-Optimal design the user can require that specific points in the
search space are included, or that others are not. Algorithms for constructing
D-Optimal designs are capable of adapting to these requirements by optimizing a
starting design. Before settling on D-Optimal designs, we explored other design
construction techniques such as the
Plackett-Burman~\cite{plackett1946design} screening designs shown in the
previous section, the \emph{contractive replacement} technique of
Addelman-Kempthorne~\cite{addelman1961some} and the \emph{direct generation}
algorithm by Grömping and Fontana~\cite{ulrike2018algorithm}. These
techniques have strong requirements on design size and level mixing, so we opted
for a more flexible technique that would enable exploring a more comprehensive
class of autotuning problems.

After the design is constructed we run each selected experiment. This step can
be done in parallel since experiments are independent. Runtime failures are
common in this step due to problems such as incorrect output. The user can
decide whether to construct a new design using the successfully completed
experiments or to continue to the analysis step if enough experiments succeed.

The next four steps of an iteration, shown in Figure \ref{fig:org2bd2283},
were discussed in detail in the previous section. User input is fundamental to
the success of these steps. After running the ANOVA test, the user should apply
domain knowledge to analyze the ANOVA table and determine which factors are
relevant. Certain factors might not appear relevant, in which case the user
should not include them in the regression model, but save them for further
exploration. Selecting the model after the ANOVA test also benefits from domain
knowledge. The impact of the number of threads used by a parallel program on its
performance is usually modeled using an inverse term, which accounts for the
speedup of adding more threads, plus a linear term, which accounts for the
overhead of their management.

A central assumption of ANOVA is the \emph{homoscedasticity} of the response, which
can be interpreted as requiring the observed error on measurements to be
independent of factor levels and of the number of measurements. Fortunately, up
to a point, there are statistical tests and corrections for lack of
homoscedasticity. Our approach uses the homoscedasticity check and correction by
power transformations from the \texttt{car} package~\cite{fox2011car} of the \texttt{R}
language before every ANOVA step.

The prediction step uses the fitted model to find factor levels that minimize
the response. The choice of the method to find these levels depends on factor
types and model and search space complexity. If factors have discrete levels,
neighborhood exploration might be needed to find valid levels that minimize the
response around the predicted levels. Validity constraints might put predicted
levels on an undefined or invalid region on the search space. This presents a
harder challenge, where the borders of valid regions would have to be explored.

The last step of an iteration is fixing factor levels to those predicted to have
best performance. The user can also decide the level of trust that will be
placed on the model and ANOVA at this step by allowing other levels. This step
performs a reduction on the dimension of the problem by eliminating factors and
decreasing the size of the search space. If we identify relevant parameters
correctly, we will have restricted further search to better regions of the
search space. In the next section we present the performance of our approach in
scenarios that differ on search space size, availability and complexity.
\section{Performance Evaluation}
\label{sec:org52ac7d9}
In this section we present performance evaluations of our approach in two
scenarios.
\vspace{-5pt}
\subsection{GPU Laplacian Kernel}
\label{sec:org08d2334}
We first evaluated the performance of our approach in a Laplacian Kernel
implemented using BOAST~\cite{videau2017boast} and targeting the \emph{Nvidia
K40c} GPU. The objective was to minimize the \emph{time to compute each pixel} by
finding the best level combination for the factors listed in Table
\ref{tab:orgc525e9f}. Considering only factors and levels, the size of the
search space is \(1.9\times10^5\) but removing points that fail at runtime yields
a search space of size \(2.3\times10^4\). The complete search space took 154 hours
to be evaluated on \emph{Debian Jessie}, using an \emph{Intel Xeon E5-2630v2} CPU,
\texttt{gcc} version \texttt{4.8.3} and \emph{Nvidia} driver version \texttt{340.32}.

We applied domain knowledge to construct the following initial performance model:
\vspace{-2pt}
\\\begin{minipage}{\linewidth}\scriptsize
\begin{align}
\label{eq:gpu_laplacian_performance_model}
\texttt{time\_per\_pixel} \sim & \scriptsize\; \texttt{y\_component\_number} + 1 / \texttt{y\_component\_number} \; + \nonumber \\
& \scriptsize\; \texttt{vector\_length} + \texttt{lws\_y} + 1 / \texttt{lws\_y} \; + \nonumber \\
& \scriptsize\; \texttt{load\_overlap} + \texttt{temporary\_size} \; + \\
& \scriptsize\; \texttt{elements\_number} + 1 / \texttt{elements\_number} \; + \nonumber \\
& \scriptsize\; \texttt{threads\_number} + 1 /\texttt{threads\_number} \nonumber
\end{align}
\vspace{2pt}
\end{minipage}
This performance model was used by the Iterative Linear Model (LM) algorithm and
by our D-Optimal Design approach (DLMT). The LM algorithm is identical to our
approach, described Section~\ref{sec:org2d608ec}, except
for the design generation step, where it uses a fixed-size random sample of the
search space instead of generating D-Optimal designs. We compared the
performance of our approach, LM and DLMT, with the following algorithms: uniform
Random Sampling (RS), Latin Hypercube Sampling (LHS), Greedy Search (GS), Greedy
Search with Restart (GSR), and Genetic Algorithm (GA), using a budget of \emph{at
most} 125 measurements, over 1000 repetitions, whithout any expert intervention.

\begin{table}[t]
\caption{\label{tab:orgc525e9f}
Parameters of the Laplacian Kernel}
\centering
\scriptsize
\begin{tabular}{ll}
\toprule
Factor & Levels\\
\midrule
\texttt{vector\_length} & \(2^0,\dots,2^4\)\\
\texttt{load\_overlap} & \textit{true}, \textit{false}\\
\texttt{temporary\_size} & \(2,4\)\\
\texttt{elements\_number} & \(1,\dots,24\)\\
\texttt{y\_component\_number} & \(1,\dots,6\)\\
\texttt{threads\_number} & \(2^5,\dots,2^{10}\)\\
\texttt{lws\_y} & \(2^0,\dots,2^{10}\)\\
\bottomrule
\end{tabular}
\end{table}

Since we measured the entire valid search space, we could use the \emph{slowdown}
relative to the \emph{global minimum} to compare the performance of algorithms.
Table~\ref{tab:gpu_laplacian_compare_budget} shows the mean, minimum and
maximum slowdowns in comparison to the global minimum, for each algorithm. It
also shows the mean and maximum budget used by each algorithm.
Figure~\ref{fig:org6b708de} presents histograms with the
real count of the slowdowns found by each of the 1000 repetitions. Arrows point
the maximum slowdown found by each algorithm.

% latex table generated in R 3.5.1 by xtable 1.8-2 package
% Mon Oct 15 18:00:57 2018
\begin{table}[ht]
\centering
\caption{Slowdown and budget used by 7 optimization methods on the Laplacian Kernel, using a budget of 125 points with 1000 repetitions}
\label{tab:gpu_laplacian_compare_budget}
\begingroup\footnotesize
\begin{tabular}{lrrrrr}
  \toprule
 & Mean & Min. & Max. & Mean Budget & Max. Budget \\
  \midrule
RS & 1.10 & 1.00 & 1.39 & 120.00 & 120.00 \\
  LHS & 1.17 & 1.00 & 1.52 & 98.92 & 125.00 \\
  GS & 6.46 & 1.00 & 124.76 & 22.17 & 106.00 \\
  GSR & 1.23 & 1.00 & 3.16 & 120.00 & 120.00 \\
  GA & 1.12 & 1.00 & 1.65 & 120.00 & 120.00 \\
  LM & 1.02 & 1.01 & 3.77 & 119.00 & 119.00 \\
   \rowcolor{red!25}DLMT & 1.01 & 1.01 & 1.01 & 54.84 & 56.00 \\
   \bottomrule
\end{tabular}
\endgroup
\end{table}

All algorithms performed relatively well in this kernel, with only Greedy Search
(GS) not being able to find slowdowns smaller than 4\(\times\) in some runs. As
expected, other search algorithms had results similar to Random Sampling (RS).
The LM algorithm was able to find the global minimum on most runs, but some runs
could not find slowdowns smaller than \(4\times\). Our approach was able to find
the global minimum in all of the 1000 runs while using \emph{at most} less than half
of the allotted budget.

We implemented a simple approach for the prediction step in this problem,
choosing the best value of our fitted models on the complete set of valid level
combinations. This was possible for this problem since all valid combinations
were known and fit in memory. For problems were the search space is too large to
be generated, we would have to either adapt this step and run the prediction on
a sample, or minimize the model using the differentiation strategies we mentioned
in Section~\ref{sec:orgd5313a5}.

\begin{figure}[t]\vspace{-.5cm}
\centering
\includegraphics[width=.9\columnwidth]{./img/comparison_histogram.pdf}
\caption{\label{fig:org6b708de}
Distribution of slowdowns in relation to the global minimum for 7 optimization methods on the Laplacian Kernel, using a budget of 125 points over 1000 repetitions \vspace{-.5cm}}
\end{figure}

This kernel provided ideal conditions for using our approach, where the
performance model is approximately known and the complete valid search space is
small enough to be stored and used for prediction. The global minimum also
appears to not be isolated in a region of points with bad performance, since our
approach was able to exploit search space geometry. We will now present a
performance evaluation of our approach in a larger and more comprehensive
benchmark.
\subsection{SPAPT Benchmark}
\label{sec:org284ec23}
The SPAPT~\cite{balaprakash2012spapt} benchmark provides parametrized CPU
kernels from different High Performance Computing domains. The kernels, shown in
Table~\ref{tab:org6b26b2f}, are implemented using the code
annotation and transformation tools provided by
Orio~\cite{hartono2009annotation}. Search space sizes are overall larger
than in the Laplacian Kernel example. Kernel factors are either integers in an
interval, such as loop unrolling and register tiling amounts, or binary flags
that control parallelization and vectorization.

We used the Random Sampling (RS) implementation available in Orio and integrated
an implementation of our approach (DLMT) to the system. Based on previous
studies~\cite{balaprakash2011can,balaprakash2012experimental} with SPAPT
kernels, we omitted the other Orio algorithms because we considered their
performance to be similar to RS. The global minima are not known for any of the
problems, and problem search spaces are too large to allow complete
measurements. Therefore, we used the performance of each application compiled
with \texttt{gcc}'s \texttt{-O3}, with no code transformations, as a \emph{baseline}
for computing the \emph{speedups} achieved by each strategy. We performed 10
autotuning repetitions for each kernel using random sampling and our approach,
using a budget of \emph{at most} 400 measurements. Our approach was also allowed to
perform only 4 of the iterations shown in Figure~\ref{fig:org2bd2283}.
Experiments were performed using Grid5000~\cite{balouek2013adding}, on
\emph{Debian Jessie}, using an \emph{Intel Xeon E5-2630v3} CPU and \texttt{gcc} version
\texttt{6.3.0}.

\begin{table}[ht]
\caption{\label{tab:org6b26b2f}
Kernels from the SPAPT benchmark used in this evaluation}
\centering
\scriptsize
\begin{tabular}{llll}
\toprule
Kernel & Operation & Factors & Size\\
\midrule
\texttt{atax} & Matrix transp. \& vector mult. & 18 & \(2.6 \times 10^{16}\)\\
\texttt{dgemv3} & Scalar, vector \& matrix mult. & 49 & \(3.8 \times 10^{36}\)\\
\texttt{gemver} & Vector mult. \& matrix add. & 24 & \(2.6 \times 10^{22}\)\\
\texttt{gesummv} & Scalar, vector, \& matrix mult. & 11 & \(5.3 \times 10^{9}\)\\
\texttt{hessian} & Hessian computation & 9 & \(3.7 \times 10^{7}\)\\
\texttt{mm} & Matrix multiplication & 13 & \(1.2 \times 10^{12}\)\\
\texttt{mvt} & Matrix vector product \& transp. & 12 & \(1.1 \times 10^{9}\)\\
\texttt{tensor} & Tensor matrix mult. & 20 & \(1.2 \times 10^{19}\)\\
\texttt{trmm} & Triangular matrix operations & 25 & \(3.7 \times 10^{23}\)\\
\texttt{bicg} & Subkernel of BiCGStab & 13 & \(3.2 \times 10^{11}\)\\
\texttt{lu} & LU decomposition & 14 & \(9.6 \times 10^{12}\)\\
\texttt{adi} & Matrix sub., mult., \& div. & 20 & \(6.0 \times 10^{15}\)\\
\texttt{jacobi} & 1-D Jacobi computation & 11 & \(5.3 \times 10^{9}\)\\
\texttt{seidel} & Matrix factorization & 15 & \(1.3 \times 10^{14}\)\\
\texttt{stencil3d} & 3-D stencil computation & 29 & \(9.7 \times 10^{27}\)\\
\texttt{correlation} & Correlation computation & 21 & \(4.5 \times 10^{17}\)\\
\bottomrule
\end{tabular}
\end{table}

The time to measure each kernel varied from a few seconds to up to 20 minutes.
We discovered in testing that some transformations caused the compiler to enter
an internal optimization process that did not stop for over 12 hours. We did not
study why these cases took so long to complete, and implemented an execution
timeout of 20 minutes, considering cases that took longer than that to compile
to be runtime failures.

Similar to the previous example, we automated factor elimination based on ANOVA
tests so that a comprehensive evaluation could be performed. We also did not
tailor initial performance models, which were the same for all kernels.
Initial models had a linear term for each factor with two or more levels, plus
quadratic and cubic terms for factors with sufficient levels. Although
automation and identical initial models might have limited the improvements at
each step of our application, our results show that it still succeeded in
decreasing the budget needed to find significant speedups for some kernels.

Figure~\ref{fig:orgd2c6a45} presents the \emph{speedup} found by each
run of RS and DLMT, plotted against the algorithm \emph{iteration} where that speedup
was found. We divided the kernels into 3 groups according to the results. The
group where no algorithm found any speedups contains 3 kernels and is marked
with \emph{blue [0]} headers. The group where both algorithms found similar speedups, in
similar iterations, contains 6 kernels and is marked with \emph{orange [=]} headers. The
group where DLMT found similar speedups using a significantly smaller budget
than RS contains 8 kernels and is marked in \emph{green [+]} headers. Ellipses delimit an
estimation of where 95\% of the underlying distribution lies, and a dashed line
marks the \texttt{-03} baseline. In comparison to RS, our approach significantly
decreased the average number of iterations needed to find speedups for the 8
kernels in the green group.

Figure~\ref{fig:org2a17872} shows the search space exploration performed
by RS and DLMT. It uses the same color groups as
Figure~\ref{fig:orgd2c6a45}, and shows the distribution, as a
count, of the speedups that where found during all repetitions of the
experiments. Histogram areas corresponding to DLMT are usually smaller because
it usually stopped earlier due to the 4-iteration limit, while RS always
performed all 400 allotted measurements. This is particularly visible in \texttt{lu},
\texttt{mvt}, \texttt{jacobi}.

We can also observe that the frequency of configurations with high speedups is
larger for DLMT overall, even for kernels on the orange group. This clearly
noticeable in \texttt{gemver, =bicgkernel}, \texttt{mm} or \texttt{tensor}. This means that our
approach spent less of the budget exploring configurations with small speedups
or slowdowns, in comparison with RS.

Finally, we can see in Figure~\ref{fig:orgd2c6a45} that the
\texttt{jacobi} kernel presents an interesting behaviour, with what appears to be two
local minima, into which both RS and DLMT may fall. Since DLMT uses a smaller
budget, we could rerun it several times and avoid getting trapped in the worst
local minimum.

Our approach used a generic initial performance model for all kernels but,
since it iteratively eliminates factors and model terms based on ANOVA tests,
was still able to exploit global search space structures for kernels in the red
and green groups. Even in this automated setting, the results with SPAPT kernels
illustrate the ability our approach has to reduce the budget needed to find good
speedups by efficiently exploring search spaces.

\begin{figure*}[p]
\centering
\includegraphics[width=\textwidth]{./img/iteration_best_comparison.pdf}
\caption{\label{fig:orgd2c6a45}
Cost of best point found on each run, and the iteration where it was found. RS and DLMT found no speedups with similar budget for kernels with \emph{blue [0]} headers, and similar speedups with similar budget for kernels with \emph{orange [=]} headers. DLMT found similar speedups with smaller budget for kernels with \emph{green [+]} headers. Ellipses delimit the 95\% confidence intervals for the averages of iterations and speedups.}
\end{figure*}

\begin{figure*}[p]
\centering
\includegraphics[width=\textwidth]{./img/split_histograms.pdf}
\caption{\label{fig:org2a17872}
Histograms of explored search spaces, showing the real count of measured configurations. Kernels are grouped in the same way as in Figure~\ref{fig:orgd2c6a45}. DLMT spent less measurements than RS in configurations with smaller speedups or with slowdowns, even for kernels in the orange group. DLMT also spent more time exploring configurations with larger speedups.}
\end{figure*}

\section{Conclusion}
\label{sec:org169f53c}
We presented in this paper a transparent Design of Experiments approach for
program autotuning under tight budget constraints. We discussed the underlying
concepts that enable our approach to significantly reduce the measurement budget
needed to find good optimizations consistently over different kernels exposing
configuration parameters of source-to-source transformations. We have made
efforts to make our results, figures and analyses reproducible by hosting all
our scripts and data publicly~\cite{bruel2018ipdps19}.

Our approach outperformed six other search heuristics, always finding the global
optimum of the search space defined by the optimization of a Laplacian kernel
for GPUs, while using at most half of the allotted budget. In a more
comprehensive evaluation using kernels from the SPAPT benchmark, our approach
was able to find the same speedups as random sampling while using up to
\(10\times\) less measurements. We showed that our approach explored search spaces
more efficiently, even for kernels where it performed similarly to random
sampling.

We presented a completely automated version of our approach in this paper so
that we could perform a thorough evaluation of its performance on comprehensive
benchmarks. In future work we will explore the impact of user input and expert
knowledge in the selection of the initial performance model and in the
subsequent elimination of factors using ANOVA tests. We expect that tailored
initial performance models and assisted factor elimination will improve the
solutions found by our approach and decrease the budget needed to find them.

Our current strategy eliminates completely from the model the factors with low
significance detected by ANOVA tests. In future work we will also explore the
effect of adding random experiments with randomized factor levels. We expect
this will decrease the impact of removing factors wrongly detected to have low
significance.

Decreasing the number of experiments needed to find optimizations is a desirable
property for autotuners in problem domains other than source-to-source
transformation. We intend to evaluate the performance of our approach in domains
such as High-Level Synthesis and compiler configuration for FPGAs, where search
spaces can get as large as \(10^{126}\), and in which we already have some
autotuning experience~\cite{bruel2017autotuninghls}.
\section*{Acknowledgment}
\label{sec:org2cc20fd}
Experiments presented in this paper were carried out using the Grid'5000
testbed, supported by a scientific interest group hosted by Inria and including
CNRS, RENATER and several Universities as well as other organizations.
This work was partly funded by CAPES, \emph{Coordenação de Aperfeiçoamento de Pessoal
de Nível Superior}, Brazil, funding code 001.

\bibliographystyle{IEEEtran}
\bibliography{references}
\end{document}
